\section{Discussion}
\label{sec:discussion}

Being able to remotely control a cockroach is useful for a couple of reasons. The first is spying and surveillance. By equipping a camera onto a roach, humans can use control to drive it and investigate a scene. Some papers have written about possible search and recovery missions, using cockroaches to find possible survivors so that humans are not looking aimlessly or putting themselves at risk when they are searching for survivors. The second reason is the simplified model. Because cockroaches have similar neurons to humans but are much less complex organisms, they can serve as a neuromechanical model system for complex organisms like humans. The principles used to steer a cockroach could be relevant to restoring function in a patient with compromised neuromuscular control. The third reason is control systems and design. What is learned about cockroach control may transfer to other robot designs or control systems. Cockroaches make use of passive mechanical feedback mechanisms to simplify the task of the neural controllers, which allows for high performance parkour. If we can understand and control a cockroach's neural interface with locomotion, we could potentially make it outperform an RC car or other bot.