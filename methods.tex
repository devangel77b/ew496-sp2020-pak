\section{Methods and Materials}

% Talk about what you plan to do in the Fall. 
% Describe so that they can be replicated. 

In the fall, I plan to use the Backyard Brains RoboRoach kit to apply electrical stimulation to a cockroach's antenna, causing it to turn in the direction opposite to the antenna with applied current. The kit includes a printed circuit board (PCB), which is a backpack that carriers the Bluetooth Low Energy wireless receiver/transmitter, which will allow communication between my smartphone and the backpack. It includes a battery for the backpack and 3 electrode sets for 3 cockroaches, each of which include one electrode for each antenna and one for ground, since electricity needs a closed circuit and the electrical stimulation current needs a return pathway. The kit also consists of all the items necessary for surgery: fine forceps (tweezers), forceps (hemostat), a loupe, a low temperature hot glue gun, hot glue sticks, Q-tips, sandpaper, loctite super glue, a popsicle stick, silly putty, dissection scissors, flour, a needle, and toothpicks. In addition to the kit, I will need a cup of ice water, paper towels, a clock, and a work lamp. I will use the Backyard Brains instructions to implement the surgery on the cockroaches.

\bigskip

I will order 2 species of cockroaches to evaluate a possible difference in performance based on species. I will order at least 5 Periplaneta americana, American cockroaches and 3 Gromphadorhina portentosa, Madagascar Hissing Cockroaches. I will keep them in separate terrariums, with plenty of food, water, and space to move around.

\bigskip

\subsection{Surgery}

The surgery alone should only take 30 to 45 minutes, but I will allot an hour for the entire experiment considering set up and clean up time. I will conduct surgery on the cockroaches one at a time. After ensuring that my work station is well lit with a work lamp, I will lay out all materials in the kit, prepare an ice bath, and load the hot glue gun with a stick of glue and plug it in, setting it on low. Next I will anesthetize the cockroach by submerging it in the ice water until it falls asleep. I will wait 2 to 5 minutes and watch for it to stop moving and reacting to stimuli such as a touch on its leg.


\subsubsection{Attach the Electrode Array}
Once the roach is fully anesthetized, I will use the forceps to carefully place the cockroach on my table. To allow the super glue to stick securely, I will sand the pronotum. With the sandpaper and without pressing too hard, I will use the hemostat forceps to grasp the pronotum and lightly sand the center of the pronotum to roughen the waxy chitin until the pronotum feels slightly rough to the touch. After, I will wipe the pronotum with a wet towel to remove any excess debris from sanding and then dry it completely with a paper towel. Next, carefully avoiding touching the glue, I will put a dab of superglue on the sanded area. With the electrodes facing towards the antennae, I will gently place the black connector on the glue, orienting it squarely with the body so that the pins are parallel to the length of the body. After waiting 1 to 2 minutes for the glue to bond to the black connector, I will place the roach back into the ice water for 1 to 2 minutes, ensuring that it is still anesthetized.


\subsubsection{Implant the Ground Electrode}
I will remove the cockroach from the ice water and place it on the table, belly down. I will carefully splay its right wing to the side, using silly putty to hold the wing down and stabilize it. Using a cotton swab, I will dry its thorax and then lightly sand the thorax in preparation for glue. Using the needle and carefully keeping away from the center line to avoid the esophagus, I will lightly poke a small hole in the exoskeleton of the thorax, just behind its head. If the roach has "freckles" on its back, I will use one of these as a reference for insertion points, which will make it easier to locate the hole when implanting the electrode. Using the fine forceps (tweezers), I will straighten the center electrode as much as possible and then carefully insert the electrode 1mm into the hole I just made. With a toothpick, I will apply a small bead of super glue to the electrode, just above where it entered the tissue. I will use the forceps to insert the electrode 1 to 3 mm into the hole, allowing the superglue to enter the body so that it will polymerize quickly and securely upon coming in contact with the internal saline. I will replace the right wing to its resting place and allow the glue to set. Once I believe it has set, I will lightly tug to test if the ground electrode is secure. Lastly, I will return the cockroach to the ice water for 1 to 2 minutes to maintain anesthetization.


\subsubsection{Implant the Right Antenna Electrode}

I will take the roach out of the ice water and lay it on its back. Using the forces to carefully splay the antenna out, I will cut the cockroach's right antenna to 3 to 6 mm. I will insert the right electrode 1 mm inside the right antenna, not all the way in. Then I will dab a bead of super glue just above where the electrode is in the antenna. I will use the forceps to insert the electrode so that the bead of glue partially enters 2 to 4 mm into the antenna. The goal is to get the glue just inside the inner ring of the antenna so the electrode will not fall out easily. I will then put the roach back into the ice water for 1 to 2 minutes to maintain anesthetization.

\subsubsection{Implant the Left Antenna Electrode}
 
This section is the reverse of the previous. I will take the roach out of the ice water and lay it on its back. Using the forces to carefully splay the antenna out, I will cut the cockroach's left antenna to 3 to 6 mm. I will insert the left electrode 1 mm inside the left antenna, not all the way in. Then I will dab a bead of super glue just above where the electrode is in the antenna. I will use the forceps to insert the electrode so that the bead of glue partially enters 2 to 4 mm into the antenna. The goal is to get the glue just inside the inner ring of the antenna so the electrode will not fall out easily. I will then put the roach back into the ice water for 1 to 2 minutes to maintain anesthetization.

\subsubsection{Tighten Wire Slack}

Because cockroach legs are strong and can pull the electrodes out if they get snagged, it is important to tighten the wire slack. Using the forceps and my fingers, I will carefully fold back the wire slack on top of the connector, trying to minimize the amount of slack wire between the antenna and connector. The exposed parts of silver wire cannot touch. I will wet and then dip the end of the popsicle stick in flour. In order to hold the excess wire in place, I will take the glue gun and place a dab of hot glue on the wires, quickly using the flour end of the popsicle stick to smush down the glue. Lastly, I will make sure that all the wires are tidy on the header, and apply extra glue to secure loose portions if necessary.

\subsubsection{End of Surgery}

Once the surgery is complete, I will place the roach back in its terrarium, giving it at least 2 to 4 hours if not a full night to recover. This surgery will be repeated for each cockroach participating in the experiment.

\subsection{Testing}

Before testing, I will download the RoboRoach app from the app store. To test my stimulation of the antennae, I will plug the male connector of the PCB into the female connector on the roach's head. I will press the small black button on the
left edge of the PCB to awaken the microcontroller. From my phone, I will use the default stimulation settings and swipe left or right to observe behavioral responses. The surgery is successful if the cockroach turns in the direction I swiped.

\bigskip

To evaluate performance, I will lay thin strips of electrical tape on the ground at angles in increments of 30 degrees, ranging from 0 to 360 degrees. I will start by placing the cockroach at 0 degrees, and then measure the angle of turn upon stimulation. 






















