\section{Introduction}
Go from broad to narrow. Why is this an interesting topic/question? Literature review stuff here. Test citation in (Author Year) format here \citep{buck2020go}. 

A literature review is appropriate here. Moore et al found that they could do fun stuff with Slovenian cockroaches \citep{moore1998directed}. A more detailed study of higher neural structures showed that blah \citep{guo2013neural}. How can engineering be used to alter or steer these biological structures? Electrical stimulation of contralateral nerve enters did blah \citep{holzer1997locomotion}. By adding external computer vision, Whitmire et al found blah \citep{whitmire2013kinect}. 

Should end with what is your guiding research question and what hypotheses are you intending to test. Based on the findings of these previous workers, I intend to test something cool... I hypothesize that... Alternatively blah... 







%PAK Scholarly Articles for Research
Tom Daniel

Robert full
Neural activity in the central complex of the cockroach brain is linked to turning behaviors 
Wired up to cockroach brain to observe neural activity
Stimulated antennae to make cockroach turn
%https://jeb.biologists.org/content/216/6/992

Simon Sponberg

Jean Mongeau
Locomotion- and mechanics-mediated tactile sensing: antenna reconfiguration simplifies control during high-speed navigation in cockroaches
When the tip of the antenna is projected backward, the animals maintain greater body-to-wall distance with fewer body collisions and less leg–wall contact than when the tip is projecting forward
Tested natural and artificial mechanosensory hairs at the tip of the antenna; both  increased an antenna's probability of switching state
%https://jeb.biologists.org/content/216/24/4530

Shai Revzen
The neuromechanics of insect locomotion: How cockroaches run fast and stably without much thought
Mathematical modeling? I’m having a hard time understanding this journal
%https://pdfs.semanticscholar.org/0479/660d2057a2ca7ac1d24a5061fcd731364ba8.pdf

Ty Hedrick

Hirotaka Sato, Christopher W. Berry, Yoav Peeri, Emen Baghoomian, Brendan E. Casey, Gabriel Lavella, John M. VandenBrooks, Jon F. Harrison and Michel M. Maharbiz
Remote radio control of insect flight
demonstrated the remote control of insects in free flight via an implantable radio-equipped miniature neural stimulating system
VERY RELEVANT BUT THEY USED BEATLES % Beatles is the band; beetles is the wildly diverse insect clade
%https://www.frontiersin.org/articles/10.3389/neuro.07.024.2009/full

Thomas E Moore, Selden B Crary, Daniel E Koditschek, Todd A Conklin
Directed Locomotion in Cockroaches: BioBots
Electrically stimulated antennae of Madagascar Hissing Cockroach to make it turn or followed prescribed paths
%https://www.zobodat.at/pdf/ActaEntSlov_6_0071-0078.pdf

R. Holzer, I. Shimoyama
Locomotion control of a bio-robotic system via electric stimulation
Electrically stimulates Periplaneta Americana to observe reactions
%https://ieeexplore.ieee.org/document/656559

Alper Bozkurt, Amit Lal, Robert Gilmour
Aerial and terrestrial locomotion control of lift assisted insect biobots
Electrically stimulates moths to steer wirelessly
%https://ieeexplore.ieee.org/document/5334433

Whitmire E, Latif T, Bozkurt A.
Kinect-based system for automated control of terrestrial insect biobots
Uses neural stimulation techniques in order to control the motion of Madagascar hissing cockroaches
%https://ieeexplore.ieee.org/document/6609789/

Link to backyard brains robo surgery: %https://backyardbrains.com/experiments/roboRoachSurgery
