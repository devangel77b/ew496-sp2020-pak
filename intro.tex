\section{Introduction}

% Why is this topic important? ?

Although the human nervous system is very different from a cockroach’s, the structure and function of our individual neurons is actually very similar. This similarity allows us to learn about the human brain by studying cockroaches'. Invertebrates, including insects like cockroaches, are important model systems in neurophysiology because of their small, manageable size, the ability to easily access specific motor neurons that innervate entire specific muscle groups, and the presence of central pattern generators that produce locomotive behaviors. 


% Literature Review
%Pak: still need to check for more of Tom Daniel's stuff with hawkmoth steering.  I put in a bunch of Bob Full/ Jean Mongeau stuff. THere are 2 things you should look up: Daniela Rus' work (I think) to steer herds of cows using miniature collars and also related work by a student? on steering flocks of runner ducks (smaller and easier to keep than cows or sheep) using machine vision, etc. I think it would be neat to relate robo roaches to those. I think I had one cow reference in my stuff but I know there's more duck stuff out there. The duck flock stuff is super cute, they have a indoor little circus ring with cameras around it and each duck has a little hat with a microcontroller and a computer vision identification tracker.

Many people have already conducted research on and tested the feasibility of steering a cockroach using electrical stimulation. \citet{moore1998directed} were able to ``drive'' Slovenian cockroaches through a zig-zagged track. They found that, although results varied, only a small fraction of the cockroaches responded and were consistently sensitive to the stimuli \citep{moore1998directed}. They also discovered that, in most cases, current directed toward the base of an antenna resulted in faster forward motion whereas current directed further down the antenna resulted in slow walking, halting, and even reverse motion \citep{moore1998directed}.

A more detailed study of higher neural structures showed that, the central complex, a part of the arthropod brain that receives sensory information and outputs to premotor regions, supervises locomotion \citep{guo2013neural}. \citet{guo2013neural} performed the same type of experiment as \citet{moore1998directed}, but also created firing maps based on the roaches' continuously changing speed and direction, which demonstrated that many of the central complex units they recorded were tuned to particular turning and forward walking speeds.

Like the previous two experiments, \citet{holzer1997locomotion} also applied electrical stimulation of nerves using an electronic backpack. Unlike the previous experiments, however, they recorded their measurements on a styrofoam trackball connected to a computer, which allowed them to record the turning rate and forward movement in response to stimulation to the roach's antennae \citep{holzer1997locomotion}. Their experiments also showed that, despite large variance, they could achieve directional locomotion control by stimulating the nerves of cockroaches' antennae.

\citet{whitmire2013kinect} demonstrated how engineering can be used to steer cockroaches by adding external computer vision, making the application of electrical stimulus autonomous. By observing successful trials of cockroach steering, the team found that the deviation from the prescribed path and the net angular change and velocity when turning could be used to create a quantitative analysis of the experiment \citep{whitmire2013kinect}. Unlike the previous study, which required a human operator to manually stimulate with a remote control, this one steered the roaches using a computer vision platform that automatically stimulates the roaches \citep{whitmire2013kinect}. The computer vision platform allowed for automated variation in stimulation patterns as the cockroach adjusted its path \citep{whitmire2013kinect}. From the quantitative analysis, the team was able to get a precise estimate of the effects of each stimulus, allowing the to optimize the efficacy of their stimulation technique \citep{whitmire2013kinect}. 


% What do I intend to do, hypothesize, etc?

Based on the findings of these previous experiments, I intend to replicate their experiments to evaluate the feasibility of a future lab exercise involving cyborg cockroaches. My ultimate aim of steering a cockroach is to learn about neurophysiological and neuromechanical principles which may be applicable in restoring human function. 

I hypothesize that modulating current to the antennae will modulate speed. If locomotion is dependent on the action of several central pattern generators being excited or inhibited, there may be limited effects on turning at the antennae and I may have to stimulate some other location to alter speed. 
Candidates for other places to stimulate are the cerci and central complex within the brain. From an engineering perspective, I want to be able to control the speed and direction of a robot. Since the previous studies have demonstrated success in direction control, my objective will be to add speed control.







