\documentclass{article}
\usepackage{graphicx} % package for displaying graphics

\usepackage[utf8]{inputenc}
\usepackage{graphicx}
\usepackage{siunitx}
%\usepackage{caption}
%\usepackage{subcaption}
%\usepackage{pdfpages}
%\usepackage{listings}
%\usepackage{soul}
%\usepackage{enumitem,amssymb}

\usepackage{color} %red, green, blue, yellow, cyan, magenta, black, white
\definecolor{mygreen}{RGB}{28,172,0} % color values Red, Green, Blue
\definecolor{mylilas}{RGB}{170,55,241}

% For Prof Evangelista to make comments
\usepackage[colorinlistoftodos]{todonotes}

% This version is also synced with a Github repository at
% https://github.com/devangel77b/ew496-sp2020-pak.git 
% and since I have the Overleaf professor edition we can also share it with
% more than two people (for example if Brooklyn Pritchard ever needs it. 

% biology style references
\usepackage[round,authoryear]{natbib}
\bibliographystyle{apalike}

\usepackage{fancyref}
% Evangelista note: should add amsmath and friends, listings if needed, hyperref for links and urls


% need a better title
\title{NEW TITLE PENDING}
\author{Alexis Pak}
\date{\today}

\begin{document}
\maketitle
\begin{abstract}
Some organisms rely on their antennae to determine the position or texture of objects around them and, more importantly, how to maneuver around them. Cockroaches (Blattidae) have tiny, hair-like sensors on their antennae that are connected to neurons that communicate messages to the brain. These messages are sent down their neural pathways in electrical activity called "spikes". This paper analyzes existing studies and determines how one could create a neural interface with a cockroach by stimulating spikes in its antennae. The product of this neural interface is the understanding of and access to the cockroaches' motor skills, steering cockroaches left or right and controlling speed, depending on the stimulation. The proximate aims of this project are to steer a cockroach, control the speed of the cockroach, and to evaluate the feasibility of a future lab exercise involving cyborg cockroaches. The ultimate aims are to learn neurophysiological and neuromechanical principles which may be applicable in restoring human function. 
\end{abstract}

%Hollywood attention grabber. 
\begin{figure}[ht!]
\centering
\includegraphics[scale=0.5]{Figures/motivation1.JPG}
\caption{Goal of experiment}
\label{fig:motivation1}
\end{figure}

\begin{figure}[ht!]
\centering
\includegraphics[scale=0.5]{Figures/motivation2.JPG}
\caption{Transformation from Cockroach to RoboRoach}
\label{fig:motivation2}
\end{figure}

% keywords: pick 3-5
{\scriptsize\textbf{Keywords:} cockroaches, Blattidae, neurophysiological, neuromechanical, steering}

\section{Introduction}

% Why is this topic important?

Although the human nervous system is very different from a cockroach’s, the structure and function of our individual neurons is actually very similar. This similarity allows us to learn about the human brain by studying cockroaches'. Invertebrates, including insects like cockroaches, are important model systems in neurophysiology because of their small, manageable size, the ability to easily access specific motor neurons that innervate entire specific muscle groups, and the presence of central pattern generators that produce locomotive behaviors. 

% Literature Review

\bigskip

Many people have already conducted research on and tested the feasibility of steering a cockroach using electrical stimulation. Moore, Crary, Koditscheck, and and Conklin were able to "drive" Slovenian cockroaches through a zig-zagged track \citep{moore1998directed}. They found that, although results varied, only a small fraction of the cockroaches responded and were consistently sensitive to the stimuli \citep{moore1998directed}. They also discovered that, in most cases, current directed toward the base of an antenna resulted in faster forward motion whereas current directed further down the antenna resulted in slow walking, halting, and even reverse motion \citep{moore1998directed}.

\bigskip

A more detailed study of higher neural structures showed that, the central complex, a part of the arthropod brain that receives sensory information and outputs to premotor regions, supervises locomotion \citep{guo2013neural}. Guo and Ritzmann performed the same type of experiment as Moore et. al., but also created firing maps based on the roaches' continuously changing speed and direction, which demonstrated that many of the central complex units they recorded were tuned to particular turning and forward walking speeds \citep{guo2013neural}.

\bigskip

Like the previous two experiments, Holzer and Shimoyama also applied electrical stimulation of nerves using an electronic backpack \citep{holzer1997locomotion}. Unlike the previous experiments, however, they recorded their measurements on a styrofoam trackball connected to a computer, which allowed them to record the turning rate and forward movement in response to stimulation to the roach's antennae \citep{holzer1997locomotion}. Their experiments also showed that, despite large variance, they could achieve directional locomotion control by stimulating the nerves of cockroaches' antennae.

\bigskip

Whitmire, Latif, and Bozkurt demonstrated how engineering can be used to steer cockroaches by adding external computer vision, making the application of electrical stimulus autonomous \citep{whitmire2013kinect}. By observing successful trials of cockroach steering, the team found that the deviation from the prescribed path and the net angular change and velocity when turning could be used to create a quantitative analysis of the experiment \citep{whitmire2013kinect}. Unlike the previous study, which required a human operator to manually stimulate with a remote control, this one steered the roaches using a computer vision platform that automatically stimulates the roaches \citep{whitmire2013kinect}. The computer vision platform allowed for automated variation in stimulation patterns as the cockroach adjusted its path \citep{whitmire2013kinect}. From the quantitative analysis, the team was able to get a precise estimate of the effects of each stimulus, allowing the to optimize the efficacy of their stimulation technique \citep{whitmire2013kinect}. 

\bigskip

% What do I intend to do, hypothesize, etc?

Based on the findings of these previous experiments, I intend to replicate their experiments to evaluate the feasibility of a future lab exercise involving cyborg cockroaches. My ultimate aim of steering a cockroach is to learn about neurophysiological and neuromechanical principles which may be applicable in restoring human function. I hypothesize that that there will be large variance in results and less than 25\% of trials will be successful, but the successful trials will demonstrate that stimulation of the left antenna will steer the cockroach right and vice versa for the opposite antenna. I also hypothesize that, for the cockroaches that do respond, more current will result in sharper and wider turns.







 %\section{Introduction}
\section{Methods and materials}
For methods talk about what you plan to do in the Fall. Describe so that they can be replicated. 

 %\section{Methods and Materials}
\section{Results}
For results talk about what you would measure.
 %\section{Results}
\section{Discussion}
\label{sec:discussion}

Being able to remotely control a cockroach is useful for a couple of reasons. The first is spying and surveillance. By equipping a camera onto a roach, humans can use control to drive it and investigate a scene. Some papers have written about possible search and recovery missions, using cockroaches to find possible survivors so that humans are not looking aimlessly or putting themselves at risk when they are searching for survivors. The second reason is the simplified model. Because cockroaches have similar neurons to humans but are much less complex organisms, they can serve as a neuromechanical model system for complex organisms like humans. The principles used to steer a cockroach could be relevant to restoring function in a patient with compromised neuromuscular control. The third reason is control systems and design. What is learned about cockroach control may transfer to other robot designs or control systems. Cockroaches make use of passive mechanical feedback mechanisms to simplify the task of the neural controllers, which allows for high performance parkour. If we can understand and control a cockroach's neural interface with locomotion, we could potentially make it outperform an RC car or other bot. %\section{Discussion}


\section{Acknowledgments}
I thank Brooklyn Pritchard, Sofia Figueroa, Cameron Smith, and John Trombetta for comments and discussion that helped my thinking on this topic; (others?)...

% References
\bibliography{pak.bib}

\appendix 
\section{Surgery}
This appendix serves to outline the exact steps to attaching the electrode array and implanting the electrodes. The surgery alone should only take \SIrange{30}{45}{\minute} , but I will allot \SI{1}{\hour} for the entire experiment considering set up and clean up time. I will conduct surgery on the cockroaches one at a time. After ensuring that my work station is well lit with a work lamp, I will lay out all materials in the kit, prepare an ice bath, and load the hot glue gun with a stick of glue and plug it in, setting it on low. Next I will anesthetize the cockroach by submerging it in the ice water until it falls asleep. I will wait \SIrange{2}{5}{\minute} and watch for it to stop moving and reacting to stimuli such as a touch on its leg.

\begin{enumerate}
\item After ensuring that my work station is well lit with a work lamp, I will lay out all materials in the kit, prepare an ice bath, and load the hot glue gun with a stick of glue and plug it in, setting it on low.
\item I will anesthetize the cockroach by submerging it in the ice water until it falls asleep.
    \subitem I will wait \SIrange{2}{5}{\minute} and watch for it to stop moving and reacting to stimuli such as a touch on its leg.
\end{enumerate}

\subsubsection{Attach the electrode array}
\begin{enumerate}
    \item Once the roach is fully anesthetized, I will use the forceps to carefully place the cockroach on my table. 
    \item To allow the super glue to stick securely, I will sand the pronotum.
    \item With the sandpaper and without pressing too hard, I will use the hemostat forceps to grasp the pronotum and lightly sand the center of the pronotum to roughen the waxy chitin until the pronotum feels slightly rough to the touch.
    \item After, I will wipe the pronotum with a wet towel to remove any excess debris from sanding and then dry it completely with a paper towel.
    \item Carefully avoiding touching the glue, I will put a dab of superglue on the sanded area.
    \item With the electrodes facing towards the antennae, I will gently place the black connector on the glue, orienting it squarely with the body so that the pins are parallel to the length of the body.
    \item After waiting \SIrange{1}{2}{\minute} for the glue to bond to the black connector, I will place the roach back into the ice water for \SIrange{1}{2}{\minute}, ensuring that it is still anesthetized.
\end{enumerate}

\subsubsection{Implant the ground electrode}
\begin{enumerate}
    \item I will remove the cockroach from the ice water and place it on the table, belly down.
    \item I will carefully splay its right wing to the side, using silly putty to hold the wing down and stabilize it.
    \item Using a cotton swab, I will dry its thorax and then lightly sand the thorax in preparation for glue.
    \item Using the needle and carefully keeping away from the center line to avoid the esophagus, I will lightly poke a small hole in the exoskeleton of the thorax, just behind its head.
        \subitem  If the roach has ``freckles'' on its back, I will use one of these as a reference for insertion points, which will make it easier to locate the hole when implanting the electrode.
    \item Using the fine forceps (tweezers), I will straighten the center electrode as much as possible and then carefully insert the electrode 1mm into the hole I just made.
    \item  With a toothpick, I will apply a small bead of super glue to the electrode, just above where it entered the tissue.
    \item I will use the forceps to insert the electrode \SIrange{1}{3}{\milli\meter} into the hole, allowing the superglue to enter the body so that it will polymerize quickly and securely upon coming in contact with the internal saline.
    \item I will replace the right wing to its resting place and allow the glue to set.
    \item Once I believe it has set, I will lightly tug to test if the ground electrode is secure.
    \item I will return the cockroach to the ice water for \SIrange{1}{2}{\minute} to maintain anesthetization.
\end{enumerate}

\subsubsection{Implant the right antenna electrode}
\begin{enumerate}
    \item I will take the roach out of the ice water and lay it on its back.
    \item Using the forces to carefully splay the antenna out, I will cut the cockroach's right antenna to \SIrange{3}{6}{\milli\meter}.
    \item I will insert the right electrode 1 mm inside the right antenna, not all the way in.
    \item I will dab a bead of super glue just above where the electrode is in the antenna.
    \item  I will use the forceps to insert the electrode so that the bead of glue partially enters \SIrange{2}{4}{\milli\meter} into the antenna.
        \subitem The goal is to get the glue just inside the inner ring of the antenna so the electrode will not fall out easily.
    \item I will then put the roach back into the ice water for \SIrange{1}{2}{\minute} to maintain anesthetization.
\end{enumerate}

\subsubsection{Implant the left antenna electrode}
This section is the reverse of the previous. 
\begin{enumerate}
    \item I will take the roach out of the ice water and lay it on its back.
    \item Using the forces to carefully splay the antenna out, I will cut the cockroach's left antenna to \SIrange{3}{6}{\milli\meter}.
    \item I will insert the left electrode \SI{1}{\milli\meter} inside the left antenna, not all the way in.
    \item I will dab a bead of super glue just above where the electrode is in the antenna.
    \item I will use the forceps to insert the electrode so that the bead of glue partially enters \SIrange{2}{4}{\milli\meter} into the antenna.
        \subitem The goal is to get the glue just inside the inner ring of the antenna so the electrode will not fall out easily.
    \item I will then put the roach back into the ice water for \SIrange{1}{2}{\minute} to maintain anesthetization.
\end{enumerate}
      
\subsubsection{Tighten wire slack}
Because cockroach legs are strong and can pull the electrodes out if they get snagged, it is important to tighten the wire slack. 
\begin{enumerate}
    \item Using the forceps and my fingers, I will carefully fold back the wire slack on top of the connector, trying to minimize the amount of slack wire between the antenna and connector. The exposed parts of silver wire cannot touch.
    \item I will wet and then dip the end of the popsicle stick in flour.
    \item In order to hold the excess wire in place, I will take the glue gun and place a dab of hot glue on the wires, quickly using the flour end of the popsicle stick to smush down the glue.
    \item I will make sure that all the wires are tidy on the header, and apply extra glue to secure loose portions if necessary.
\end{enumerate}

\subsubsection{End of surgery}
Once the surgery is complete, I will place the roach back in its terrarium, giving it at least \SIrange{2}{4}{\hour} if not a full night to recover. This surgery will be repeated for each cockroach participating in the experiment.

 % Detailed roach surgery as appendix
\end{document}









