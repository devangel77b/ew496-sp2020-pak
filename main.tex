\documentclass{article}

\usepackage[utf8]{inputenc}
\usepackage{natbib}
\usepackage{graphicx}
%\usepackage{caption}
%\usepackage{subcaption}
%\usepackage{pdfpages}
%\usepackage{listings}
%\usepackage{soul}
%\usepackage{enumitem,amssymb}

\usepackage{color} %red, green, blue, yellow, cyan, magenta, black, white
\definecolor{mygreen}{RGB}{28,172,0} % color values Red, Green, Blue
\definecolor{mylilas}{RGB}{170,55,241}

% For Prof Evangelista to make comments
\usepackage[colorinlistoftodos]{todonotes}

% This version is also synced with a Github repository at
% https://github.com/devangel77b/ew496-sp2020-pak.git 
% and since I have the Overleaf professor edition we can also share it with
% more than two people (for example if Brooklyn Pritchard ever needs it. 



\title{RoboRoach Report EW496}
\author{Alexis Pak}
\date{April 2020}

\begin{document}
\maketitle
\begin{abstract}
Some organisms rely on their antennae to determine the position or texture of objects around them and, more importantly, how to maneuver around them. Cockroaches have tiny, hair-like sensors on their antennae that are connected to neurons that communicate messages to the brain. These messages are sent down their neural pathways in electrical activity called "spikes". This paper analyzes existing studies and determines how one could create a neural interface with a cockroach by stimulating spikes in its antennae. The product of this neural interface is understanding of and access to the cockroach's motor skills, making the cockroach turn left or right depending on the stimulation. 
\end{abstract}

DIRECTION OF PAPER: I THINK I WILL EVALUATE THE DIFFERENT METHODS AND RESULTS AND RECOMMEND THE MOST EFFECTIVE WAY?


\section{Introduction}
\section{Methods and Materials}
\section{Results}
\section{Discussion}
\section{Acknowledgments}

% references go here
%\biblography{pak.bib}
\end{document}









